\chapter{Load Balancer and Virtual Host Setup}
\label{chap:lbAndVH}

% =========================================================
\section{A — Project: Load Balancer}

\subsection{File Structure}
\begin{minted}[fontsize=\small]{text}
load-balanced-app/
├── docker-compose.yml
├── nginx/
│   └── nginx.conf
├── web1/
│   └── index.html
└── web2/
    └── index.html
\end{minted}

\subsection{File Contents}

\noindent\textbf{web1/index.html}
\begin{minted}[fontsize=\small]{html}
<h1>Hello from Web 1</h1>
\end{minted}

\noindent\textbf{web2/index.html}
\begin{minted}[fontsize=\small]{html}
<h1>Hello from Web 2</h1>
\end{minted}

\noindent\textbf{nginx/nginx.conf}
\begin{minted}[fontsize=\small]{nginx}
events {}

http {
    upstream backend {
        server web1:80;
        server web2:80;
    }

    server {
        listen 80;

        location / {
            proxy_pass http://backend;
            proxy_set_header Host $host;
            proxy_set_header X-Real-IP $remote_addr;
        }
    }
}
\end{minted}

\noindent\textbf{docker-compose.yml}
\begin{minted}[fontsize=\small]{yaml}
version: '3'

services:
  web1:
    image: nginx
    container_name: web1
    volumes:
      - ./web1:/usr/share/nginx/html:ro

  web2:
    image: nginx
    container_name: web2
    volumes:
      - ./web2:/usr/share/nginx/html:ro

  loadbalancer:
    image: nginx
    container_name: loadbalancer
    ports:
      - "8080:80"
    volumes:
      - ./nginx/nginx.conf:/etc/nginx/nginx.conf:ro
    depends_on:
      - web1
      - web2
\end{minted}

\subsection{Steps Executed}
\begin{minted}[fontsize=\small]{bash}
cd /path/to/load-balanced-app
docker-compose up -d --build
docker ps
# verify containers: web1, web2, loadbalancer

# local browser test:
# open http://localhost:8080 and refresh to see alternating servers

docker-compose down
\end{minted}

% =========================================================
\section{B — Project: Virtual Host (Name-Based)}

\subsection{File Structure}
\begin{minted}[fontsize=\small]{text}
virtual-hosts/
├── docker-compose.yml
├── nginx/
│   └── nginx.conf
├── web1/
│   └── index.html
└── web2/
    └── index.html
\end{minted}

\subsection{File Contents}

\noindent\textbf{web1/index.html}
\begin{minted}[fontsize=\small]{html}
<h1>This is Web 1 Virtual Host</h1>
\end{minted}

\noindent\textbf{web2/index.html}
\begin{minted}[fontsize=\small]{html}
<h1>This is Web 2 Virtual Host</h1>
\end{minted}

\noindent\textbf{nginx/nginx.conf}
\begin{minted}[fontsize=\small]{nginx}
events {}

http {
    server {
        listen 80;
        server_name web1.domain.com;

        location / {
            proxy_pass http://web1:80;
            proxy_set_header Host $host;
            proxy_set_header X-Real-IP $remote_addr;
        }
    }

    server {
        listen 80;
        server_name web2.domain.com;

        location / {
            proxy_pass http://web2:80;
            proxy_set_header Host $host;
            proxy_set_header X-Real-IP $remote_addr;
        }
    }
}
\end{minted}

\noindent\textbf{docker-compose.yml}
\begin{minted}[fontsize=\small]{yaml}
version: '3'

services:
  web1:
    image: nginx
    container_name: vh_web1
    volumes:
      - ./web1:/usr/share/nginx/html:ro

  web2:
    image: nginx
    container_name: vh_web2
    volumes:
      - ./web2:/usr/share/nginx/html:ro

  proxy:
    image: nginx
    container_name: vh_proxy
    ports:
      - "80:80"
    volumes:
      - ./nginx/nginx.conf:/etc/nginx/nginx.conf:ro
    depends_on:
      - web1
      - web2
\end{minted}

\subsection{Steps Executed}
\begin{minted}[fontsize=\small]{bash}
cd /path/to/virtual-hosts
docker-compose up -d --build
docker ps
# verify containers: vh_web1, vh_web2, vh_proxy

curl -H "Host: web1.domain.com" http://localhost
curl -H "Host: web2.domain.com" http://localhost

docker-compose down
\end{minted}

% =========================================================
\section{C — Local DNS (Simulated Domains)}

\subsection{Windows Hosts File Configuration}
\begin{minted}[fontsize=\small]{text}
C:\Windows\System32\drivers\etc\hosts
\end{minted}

\begin{minted}[fontsize=\small]{text}
142.93.249.208 web1.domain.com
142.93.249.208 web2.domain.com
\end{minted}
Here the IP Address before the domain is the <digital-ocean-droplet-ipv4-addr>

\begin{minted}[fontsize=\small]{bash}
ipconfig /flushdns
\end{minted}

Open in browser:
\begin{itemize}
    \item http://web1.domain.com
    \item http://web2.domain.com
\end{itemize}

\subsection{Linux / Unix Hosts File Configuration}
\begin{minted}[fontsize=\small]{bash}
sudo nano /etc/hosts
\end{minted}

Add:
\begin{minted}[fontsize=\small]{text}
142.93.249.208 web1.domain.com
142.93.249.208 web2.domain.com
\end{minted}
Here the IP Address before the domain is the <digital-ocean-droplet-ipv4-addr>

Save and apply:
\begin{minted}[fontsize=\small]{bash}
sudo systemctl restart NetworkManager
# or
sudo dscacheutil -flushcache  # macOS
\end{minted}

Test:
\begin{minted}[fontsize=\small]{bash}
ping web1.domain.com
ping web2.domain.com
\end{minted}

Open in browser:
\begin{itemize}
    \item http://web1.domain.com
    \item http://web2.domain.com
\end{itemize}

\section{Executed Outputs}

\begin{figure}[h]
    \centering
    \includegraphics[width=0.8\textwidth]{png/loadBalancerAndVirtualHost/LB-1.png}
    \caption{Load Balancer Response 1}
    \label{fig:lbr1}
\end{figure}

\begin{figure}[htb]
    \centering
    \includegraphics[width=0.75\textwidth]{png/loadBalancerAndVirtualHost/LB-2.png}
    \caption{Load Balancer Response 2}
    \label{fig:lbr2}
\end{figure}

\begin{figure}[t]
    \centering
    \includegraphics[scale=0.9]{png/loadBalancerAndVirtualHost/VH-1.png}
    \caption{Virtual Host Response 1}
    \label{fig:vhr1}
\end{figure}

\begin{figure}[ht]
    \centering
    \includegraphics[height=5cm]{png/loadBalancerAndVirtualHost/VH-2.png}
    \caption{Virtual Host Response 2}
    \label{fig:vhr2}
\end{figure}

\begin{figure}[ht]
    \centering
    \includegraphics[height=5cm]{png/loadBalancerAndVirtualHost/droplet.png}
    \caption{Inside Droplet 1}
    \label{fig:drop-1}
\end{figure}

\begin{figure}[ht]
    \centering
    \includegraphics[height=5cm]{png/loadBalancerAndVirtualHost/droplet-2.png}
    \caption{Inside Droplet 2}
    \label{fig:drop-2}
\end{figure}

\begin{figure}[ht]
    \centering
    \includegraphics[height=5cm]{png/loadBalancerAndVirtualHost/docker.png}
    \caption{Docker Processes running}
    \label{fig:docker-ps}
\end{figure}