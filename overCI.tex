\chapter{Overleaf CI with GitHub Action}
\label{chap:overleafCI}

\section{Introduction}

This chapter documents the implementation of a CI/CD\index{CI/CD} pipeline using GitHub Actions\index{GitHub Actions} to automatically compile LaTeX\index{LaTeX} documents, inject version information, and archive compiled PDFs.

\section{Project Setup from Overleaf}

\subsection{Exporting Overleaf Project}

The collaborative LaTeX project was initially developed on Overleaf\index{Overleaf}. The project was exported and prepared for Git version control\index{version control}:

\begin{minted}[breaklines,fontsize=\small]{bash}
# Download project from Overleaf as ZIP file
# File downloaded: overleaf-project.zip

# Extract the ZIP file
unzip overleaf-project.zip -d latex-project
cd latex-project

# Initialize Git repository
git init

# Create initial commit
git add .
git commit -m "Initial commit: Import from Overleaf"

# Create GitHub repository and link
git remote add origin https://github.com/Badri-Narayanan/overleaf-local-grp14.git
git branch -M main
git push -u origin main
\end{minted}

\subsection{Project Structure}

After extraction, the project contains:

\begin{minted}[breaklines,fontsize=\small]{text}
latex-project/
├── groupAssignments.tex        # Main document
├── itGlossary.tex              # Glossary definitions
├── bibfile.bib                 # Bibliography
├── history.tex                 # Revision history
├── itPasswords.tex             # Service credentials
├── linuxCommands.tex           # Chapter 1
├── projectProposal.tex         # Chapter 2
├── awsDeployment.tex           # Chapter 3
├── digitalOceanSetup.tex       # Chapter 4
├── overleafDeployment.tex      # Chapter 5
└── png/                        # Images directory
\end{minted}

\section{GitHub Actions Workflow Setup}

\subsection{Workflow File Creation}

Create \texttt{.github/workflows/label-project.yml}\index{workflow file}:

\begin{minted}[breaklines,fontsize=\small]{yaml}
name: Build LaTeX PDF

permissions:
  contents: write

on:
  push:
    branches:
      - main
  workflow_dispatch:

jobs:
  build:
    runs-on: ubuntu-latest

    steps:
      - name: Checkout repository
        uses: actions/checkout@v4
        with:
          persist-credentials: true
          fetch-depth: 0

      - name: Install TeX Live and Pygments
        run: |
          sudo apt-get update
          sudo apt-get install -y texlive-full python3-pygments

      - name: Get Git commit hash
        run: echo "GIT_HASH=$(git rev-parse --short HEAD)" >> $GITHUB_ENV

      - name: Update version in groupAssignments.tex
        run: |
          sed -i "s/\\\\newcommand{\\\\gitversion}{Unknown}/\\\\newcommand{\\\\gitversion}{${GIT_HASH}}/" groupAssignments.tex
          echo "Updated version to: ${GIT_HASH}"
          grep "gitversion" groupAssignments.tex

      - name: Compile LaTeX document (Pass 1)
        run: |
          pdflatex -shell-escape -interaction=nonstopmode groupAssignments.tex || true

      - name: Run BibTeX
        run: |
          bibtex groupAssignments || true

      - name: Run MakeIndex
        run: |
          makeindex groupAssignments.idx || true

      - name: Compile LaTeX document (Pass 2)
        run: |
          pdflatex -shell-escape -interaction=nonstopmode groupAssignments.tex || true

      - name: Compile LaTeX document (Pass 3 - Final)
        run: |
          pdflatex -shell-escape -interaction=nonstopmode groupAssignments.tex || true

      - name: Check if PDF was generated
        run: |
          if [ -f "groupAssignments.pdf" ]; then
            echo "PDF generated successfully!"
            ls -lh groupAssignments.pdf
          else
            echo "ERROR: PDF was not generated!"
            exit 1
          fi

      - name: Upload PDF artifact
        uses: actions/upload-artifact@v4
        with:
          name: Compiled-PDF
          path: groupAssignments.pdf

      - name: Commit PDF to repo
        if: success()
        env:
          GIT_HASH: ${{ env.GIT_HASH }}
        run: |
          git config user.name "Badri-Narayanan"
          git config user.email "badhrirajen@gmail.com"
          mkdir -p builds
          cp groupAssignments.pdf builds/document_${GIT_HASH}.pdf
          git add builds/document_${GIT_HASH}.pdf
          git commit -m "Auto-build PDF for commit ${GIT_HASH}" || echo "No changes to commit"
          git push origin HEAD:main || echo "Push failed"
\end{minted}

\subsection{Key Workflow Components}

\textbf{Critical Flags:}
\begin{itemize}
    \item \texttt{-shell-escape}: Required for minted package\index{shell-escape}
    \item \texttt{-interaction=nonstopmode}: Non-interactive compilation
    \item \texttt{|| true}: Continue on warnings
\end{itemize}

\textbf{Dependencies:}
\begin{itemize}
    \item \texttt{texlive-full}: Complete LaTeX distribution
    \item \texttt{python3-pygments}: Required for minted syntax highlighting\index{Pygments}
\end{itemize}

\section{LaTeX Document Configuration}

\subsection{Preamble Setup}

Modify \texttt{groupAssignments.tex} preamble:

\begin{minted}[breaklines,fontsize=\small]{latex}
\documentclass[12pt,a4paper]{report}
\usepackage[utf8]{inputenc}
\usepackage[margin=1in]{geometry}
\usepackage{titling}
\usepackage{array}
\usepackage{booktabs}
\usepackage{fancyhdr}
\usepackage{minted}
\usepackage{graphicx}
\usepackage{url}
\usepackage{longtable}
\usepackage{enumitem}
\usepackage{makeidx}

% Hyperref configuration (removed pdfmark option)
\usepackage[
breaklinks=true, 
colorlinks=true,
citecolor=blue,
linkcolor=blue,
menucolor=black,
urlcolor=blue
]{hyperref}

% Glossaries configuration (removed unsupported options)
\usepackage[toc]{glossaries}

% Load glossary definitions
\input{itGlossary.tex}
\makenoidxglossaries

% Make index
\makeindex

% Git version command - updated by GitHub Actions
\newcommand{\gitversion}{Unknown}

% Header/Footer setup
\pagestyle{fancy}
\fancyhf{}
\fancyfoot[C]{\thepage}
\fancyhead[L]{v\gitversion}
\fancyhead[C]{Chapter \thechapter \hspace{1em} \copyright Stevens -- \today \hspace{1em} -- Do Not Distribute!}
\setlength{\headheight}{15pt}
\renewcommand{\headrulewidth}{0pt}
\end{minted}

\subsection{Title Page with Version}

\begin{minted}[breaklines,fontsize=\small]{latex}
\begin{titlepage}
    \centering
    \vspace*{5cm}
    
    {\Huge\bfseries DevOps - Group 14 Assignments\par}
    
    \vspace{0.5cm}
    {\large Document Version: \gitversion\par}
    
    \vspace{1cm}
    
    {\Large by\par}
    
    \vspace{0.5cm}
    
    {\Large
    Andre Santiago-Neyra\\
    Badri Narayanan Rajendran\\
    }
    
    \vspace{1cm}
    
    {\large Stevens.edu\par}
    
    \vfill
    
    {\large October 2, 2025\par}
    
\end{titlepage}
\end{minted}

\section{Glossary Configuration}

\subsection{Adding Glossary Entries}

In \texttt{itGlossary.tex}, add required entries:

\begin{minted}[breaklines,fontsize=\small]{latex}
\newglossaryentry{cicd}{
    name={CI/CD},
    description={Continuous Integration and Continuous Deployment - automated software development practices for frequent code integration and deployment}
}
\end{minted}

\subsection{Using Glossary Entries}

In document text:

\begin{minted}[breaklines,fontsize=\small]{latex}
... deployment processes using two \gls{cicd} tools: Jenkins and GitHub Actions...
\end{minted}

\section{Implementation Steps}

\subsection{Step 1: Export and Setup from Overleaf}

\begin{minted}[breaklines,fontsize=\small]{bash}
# Download ZIP from Overleaf
# Extract and initialize repository
unzip overleaf-project.zip -d latex-project
cd latex-project
git init
git add .
git commit -m "Initial commit from Overleaf"
\end{minted}

\subsection{Step 2: Create GitHub Repository}

\begin{minted}[breaklines,fontsize=\small]{bash}
# Create repository on GitHub (via web interface)
# Link local repository to GitHub
git remote add origin https://github.com/Badri-Narayanan/overleaf-local-grp14.git
git branch -M main
git push -u origin main
\end{minted}

\subsection{Step 3: Create Workflow File}

\begin{minted}[breaklines,fontsize=\small]{bash}
# Create workflow directory structure
mkdir -p .github/workflows

# Create workflow file
touch .github/workflows/label-project.yml

# Add workflow YAML content (as shown in Section 3.1)
\end{minted}

\subsection{Step 4: Update LaTeX Preamble}

Modifications to \texttt{groupAssignments.tex}:

\begin{enumerate}
    \item Remove \texttt{pdfmark} from hyperref options
    \item Change glossaries package: \texttt{\textbackslash usepackage[toc]\{glossaries\}}
    \item Add version command: \texttt{\textbackslash newcommand\{\textbackslash gitversion\}\{Unknown\}}
    \item Set header height: \texttt{\textbackslash setlength\{\textbackslash headheight\}\{15pt\}}
    \item Add version to title page and header
\end{enumerate}

\subsection{Step 5: Fix Glossary Issues}

\begin{minted}[breaklines,fontsize=\small]{bash}
# Edit itGlossary.tex
# Add missing glossary entry for 'cicd'
\end{minted}

Add to \texttt{itGlossary.tex}:
\begin{minted}[breaklines,fontsize=\small]{latex}
\newglossaryentry{cicd}{
    name={CI/CD},
    description={Continuous Integration and Continuous Deployment}
}
\end{minted}

\subsection{Step 6: Commit and Push Changes}

\begin{minted}[breaklines,fontsize=\small]{bash}
# Stage all changes
git add .github/workflows/label-project.yml
git add groupAssignments.tex
git add itGlossary.tex

# Commit changes
git commit -m "Add GitHub Actions CI/CD for LaTeX compilation"

# Push to trigger workflow
git push origin main
\end{minted}

\subsection{Step 7: Monitor Workflow Execution}

\begin{enumerate}
    \item Navigate to GitHub repository
    \item Click "Actions" tab
    \item View real-time workflow execution
    \item Wait for green checkmark (3-5 minutes)
\end{enumerate}

\subsection{Step 8: Access Compiled PDF}

\textbf{Method 1 - Workflow Artifacts:}
\begin{minted}[breaklines,fontsize=\small]{bash}
# Via GitHub web interface:
# Actions → Select workflow run → Artifacts → Download "Compiled-PDF"
\end{minted}

\textbf{Method 2 - Repository Builds Directory:}
\begin{minted}[breaklines,fontsize=\small]{bash}
# Navigate to builds/ directory in repository
# Access: builds/document_[commit-hash].pdf
# Example: builds/document_f3f37c0.pdf
\end{minted}
